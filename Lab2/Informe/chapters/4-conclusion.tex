\section{Conclusiones Globales}

\subsection{Parte I}

\par Como se pudo evidenciar en la parte I, los resultados obtenidos sobre el cálculo de la integral fueron muy precisos en los tres métodos utilizados, dándonos así el mismo valor para los 3 resultados. A pesar de que todos los métodos cumplen con el objetivo constan con una efectividad distintas debido a su manera de atacar las funciones, a excepción se puede decir del método de trapecio simples, que como se pudo apreciar en la tercera prueba, cuando se tiene una tolerancia no tan pequeña, el método no dan exactamente al resultado con un error mucho mayor. 

\subsection{Parte II}

\par  Para la parte II los resultados fueron variados. Primero se logró implementar bien el método de diferencias finitas en la ecuación diferencial parcial, ya que el gráfico de dispersión tiempo-espacio tiene gran similitud al original. Se debe destacar el número del error y el costo computacional, los cuales podían haber sido mejores.

\par La mayor dificultad fue obtener las curvas de elución y modulación. Para la curva de elución no hubo problema, ya que como se pudo apreciar en ambas pruebas se pudo lograr, pero la curva del modelamiento representa varias fallas en la parte intermedia de la curva (cuando asciende), aunque en la parte inicial y final de la curva se logra su equilibrio. Así que para esta parte del laboratorio se puede decir que se logró casi todo.

\par Mencionar que se cumplió con el objetivo principal del laboratorio que
era crear los algoritmos y analizar cada uno de sus resultados, lo cual al terminar con éxito
este laboratorio nos entregó herramientas y mayor entendimiento de cómo implementar una
abstracción matemática a un computador.

