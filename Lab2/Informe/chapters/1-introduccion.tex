\section{Introducción}

\par En la actualidad todo gira en torno a los números y diversos cálculos matemáticos, para esto se realizan distintos tipos de operaciones. Por lo mencionado anteriormente, es que se usa una rama de la matemática llamada análisis numérico, que mediante distintos tipos de métodos numéricos se realizan cálculos y a su vez un completo análisis de estos resultados obtenidos.

\par Un buen uso de los métodos numéricos otorga una infinidad de habilidades, las cuales pueden ser ocupadas en el día a día. Sin embargo, el problema que ocurre es que la computadora no puede aprender ciertas habilidades, por el simple motivo de no tener la capacidad pensante del humano. Debido a esto es que se ocupan técnicas para disminuir la dificultad de los distintos métodos numéricos, algunas  de las más utilizadas es la aproximación.

\par El desarrollo del presente laboratorio se divide en dos partes las cuales son categorizadas como: Calculo de una integral definida mediante 3 formas de calculo para esta, y la aplicacion de un método o tecnica para resolver una ecuacion difernecial parcial.

\subsection{Objetivos}

\par El objetivo que busca dicho laboratorio, es aplicar cada uno de los métodos de calculo de la intregal para la Parte I, y resolver con algun metodo la ecuacion diferencial parcial de Cromatografia de Membrana para la Parte II. Para ello se deben diseñar algoritmos capaces de entregarmel resultado aproximado de los calculos.

\subsection{Herramientas}

\par Las herramientas ocupadas para el desarrollo del laboratorio e informe fueron las siguientes.

\begin{enumerate}
	\item Para la realización del código fuente de ambas partes se ocupó Matlab R2918b.
	\item  Para la realización de la documentación del laboratorio se ocupó Latex-Texstudio 
\end{enumerate}



 